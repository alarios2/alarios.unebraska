\documentclass{beamer}
\usepackage{beamerthemeshadow,multicol,tikz,enumerate}
% \usetheme{UNLTheme}
\usetheme{CambridgeUS} %Useing this theme for more room on the screen.



\begin{document}

\title[Inverse Functions]{Calculus 1 \\ The Chain Rule and Inverse Functions}  
% \author{Dr. Somebody}
\date{} 

% vvvvvvvvvvvvvvvvvvvvvvvvvvvvvvvvvvvvvvvvvvvvvvvvvvvvvvvvvvvv
\begin{frame}{Clicker Survey}

How do you feel about the first test?

\begin{enumerate}[(a)]
 \item Great
 \item Good
 \item Average
 \item No so good
 \item Terrible
\end{enumerate}

 
\end{frame}
% ^^^^^^^^^^^^^^^^^^^^^^^^^^^^^^^^^^^^^^^^^^^^^^^^^^^^^^^^^^^^
% vvvvvvvvvvvvvvvvvvvvvvvvvvvvvvvvvvvvvvvvvvvvvvvvvvvvvvvvvvvv
\begin{frame}{Clicker Survey}

Do you think you need to:

\begin{enumerate}[(a)]
\item Study about the same for the next test

\item Study harder for the next test

\item Study less for the next test

\item Study Sooner for the next test
\end{enumerate}
 
\end{frame}
% ^^^^^^^^^^^^^^^^^^^^^^^^^^^^^^^^^^^^^^^^^^^^^^^^^^^^^^^^^^^^
% vvvvvvvvvvvvvvvvvvvvvvvvvvvvvvvvvvvvvvvvvvvvvvvvvvvvvvvvvvvv
\begin{frame}{}

We have already used inverse functions to solve things such as $\tan(\arcsin(3/5))=$?

We now wish to take advantage of the fact that \[f(f^{-1}(x))=x\] to find some derivatives of new functions.

Let's practice the idea on something we already know: 
\[\text{If }f(x) = \sqrt{x},\text{ then }(f(x))^2 = x.\]
Take the derivative of both sides:
\[2(f(x))\frac{df}{dx} = 1\]
\[\text{Therefore,}\qquad\frac{df}{dx} = \frac{1}{2f(x)} = \frac{1}{2\sqrt{x}}\]
as expected.
\end{frame}
% ^^^^^^^^^^^^^^^^^^^^^^^^^^^^^^^^^^^^^^^^^^^^^^^^^^^^^^^^^^^^
% vvvvvvvvvvvvvvvvvvvvvvvvvvvvvvvvvvvvvvvvvvvvvvvvvvvvvvvvvvvv
\begin{frame}{The Derivative of $\ln(x)$}
 We use the chain rule to differentiate an identity involving $\ln x$.  
 
 Since $e^{\ln x}  = x$, we can differentiate both sides.  On the one hand we have:
\begin{align*}
 \frac{d}{dx}e^{\ln x} = \frac{d}{dx} x = 1.
\end{align*}
Also, by the chain rule, 
\begin{align*}
 \frac{d}{dx}e^{\ln x} = e^{\ln x} \frac{d}{dx}\ln(x) = x \frac{d}{dx}\ln(x)
\end{align*}
Thus, dividing by $x$,
\begin{align*}
 \frac{d}{dx}\ln x = \frac1x
\end{align*}



\end{frame}
% ^^^^^^^^^^^^^^^^^^^^^^^^^^^^^^^^^^^^^^^^^^^^^^^^^^^^^^^^^^^^
% vvvvvvvvvvvvvvvvvvvvvvvvvvvvvvvvvvvvvvvvvvvvvvvvvvvvvvvvvvvv
\begin{frame}{}
 
\end{frame}
% ^^^^^^^^^^^^^^^^^^^^^^^^^^^^^^^^^^^^^^^^^^^^^^^^^^^^^^^^^^^^
% vvvvvvvvvvvvvvvvvvvvvvvvvvvvvvvvvvvvvvvvvvvvvvvvvvvvvvvvvvvv
\begin{frame}{}
 
\end{frame}
% ^^^^^^^^^^^^^^^^^^^^^^^^^^^^^^^^^^^^^^^^^^^^^^^^^^^^^^^^^^^^
% vvvvvvvvvvvvvvvvvvvvvvvvvvvvvvvvvvvvvvvvvvvvvvvvvvvvvvvvvvvv
\begin{frame}{}
 
\end{frame}
% ^^^^^^^^^^^^^^^^^^^^^^^^^^^^^^^^^^^^^^^^^^^^^^^^^^^^^^^^^^^^
% vvvvvvvvvvvvvvvvvvvvvvvvvvvvvvvvvvvvvvvvvvvvvvvvvvvvvvvvvvvv
\begin{frame}{}
 
\end{frame}
% ^^^^^^^^^^^^^^^^^^^^^^^^^^^^^^^^^^^^^^^^^^^^^^^^^^^^^^^^^^^^
% vvvvvvvvvvvvvvvvvvvvvvvvvvvvvvvvvvvvvvvvvvvvvvvvvvvvvvvvvvvv
\begin{frame}{}
 
\end{frame}
% ^^^^^^^^^^^^^^^^^^^^^^^^^^^^^^^^^^^^^^^^^^^^^^^^^^^^^^^^^^^^
% vvvvvvvvvvvvvvvvvvvvvvvvvvvvvvvvvvvvvvvvvvvvvvvvvvvvvvvvvvvv
\begin{frame}{}
 
\end{frame}
% ^^^^^^^^^^^^^^^^^^^^^^^^^^^^^^^^^^^^^^^^^^^^^^^^^^^^^^^^^^^^
% vvvvvvvvvvvvvvvvvvvvvvvvvvvvvvvvvvvvvvvvvvvvvvvvvvvvvvvvvvvv
\begin{frame}{}
 
\end{frame}
% ^^^^^^^^^^^^^^^^^^^^^^^^^^^^^^^^^^^^^^^^^^^^^^^^^^^^^^^^^^^^
% vvvvvvvvvvvvvvvvvvvvvvvvvvvvvvvvvvvvvvvvvvvvvvvvvvvvvvvvvvvv
\begin{frame}{}
 
\end{frame}
% ^^^^^^^^^^^^^^^^^^^^^^^^^^^^^^^^^^^^^^^^^^^^^^^^^^^^^^^^^^^^
% vvvvvvvvvvvvvvvvvvvvvvvvvvvvvvvvvvvvvvvvvvvvvvvvvvvvvvvvvvvv
\begin{frame}{}
 
\end{frame}
% ^^^^^^^^^^^^^^^^^^^^^^^^^^^^^^^^^^^^^^^^^^^^^^^^^^^^^^^^^^^^
% vvvvvvvvvvvvvvvvvvvvvvvvvvvvvvvvvvvvvvvvvvvvvvvvvvvvvvvvvvvv
\begin{frame}{}
 
\end{frame}
% ^^^^^^^^^^^^^^^^^^^^^^^^^^^^^^^^^^^^^^^^^^^^^^^^^^^^^^^^^^^^


\end{document}
