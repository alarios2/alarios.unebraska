\documentclass[]{article}


\date{10/14/16}

\begin{document}

Dear Graduate Students,

Here is some disappointing news: Much of science relies on solving differential equations, and most differential equations cannot be solved by the human race.  Now, here is some exciting news: if we allow for *approximate* solutions, we can solve almost all differential equations!  This is one of the things that makes the Spring 2017 course:

  Math 934 - Topics in Differential Equations

so great.

We will solve differential equations using numerical methods, and this does not mean that we will leave mathematical rigor or beauty behind!  The subject of numerical PDEs is full of clever ideas, elegant structures, dazzling schemes, and subtle concepts.  The world of PDEs is so vast, that you can spend several lifetimes studying just one PDE, and yet there are thousands of PDEs out there.  They are used to model phenomena such as weather, turbulence, blood flow, cancer growth, traffic, financial markets, ecology, acoustics, electricity, magnetism, star formation, and the bending of spacetime itself.  Solving them not only unlocks new areas of science, but often leads to pretty pictures that can amaze people in your poster sessions and astonish audiences at conferences.

To set the stage, we will begin with numerical solutions of ODEs (ordinary differential equations).  We will quickly move on to numerical solutions of PDEs (partial differential equations).  We will learn spectral/Fourier methods, finite element methods, finite volume methods, saddle-point methods, and, time permitting, several other methods.  We will learn to program in Matlab, and we will also also a finite element library called FEniCS, which uses Python.

No programming background is necessary, and it is not necessary for you to have taken a course in PDEs.  An undergraduate ODE course, such as Math 221, advanced knowledge of linear algebra, such as Math 415/815 (or anything beyond Math 314), and some analysis, such as Math 825/826 (could be taken concurrently), should be sufficient.

I hope you will join us this spring!

\end{document}
