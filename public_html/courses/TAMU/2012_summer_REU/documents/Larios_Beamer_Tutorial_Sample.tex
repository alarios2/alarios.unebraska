% $Header: /cvsroot/latex-beamer/latex-beamer/solutions/conference-talks/conference-ornate-20min.en.tex,v 1.6 2004/10/07 20:53:08 tantau Exp $

\documentclass[leqno]{beamer}
%\documentclass[handout,leqno,hyperref={pdfpagelabels=false}]{beamer}
%\documentclass[draft,leqno,hyperref={pdfpagelabels=false}]{beamer}
%% Note: ``hyperref={pdfpagelabels=false}'' is used to get rid of
%% an unnecessary LaTeX warning.  Turn it off in modern versions.

%% This gets rid of a bug in Beamer:
\let\Tiny=\tiny

\usepackage{amsmath,amsfonts,amssymb,mathrsfs}
\usepackage{enumerate}
\usepackage{color}
\usepackage{graphicx}

%% ==== Beamer Specific Packages and Options ====
%% Use a more LaTeX-like font in math environments:
\usefonttheme[onlymath]{serif} 

%% Enable the ability to use hyperlinks:
\usepackage{hyperref}

%This gets rid of another Beamer compilation warning bug.
% \usepackage{lmodern} 

%% ==== Templates ====
%\setbeamertemplate{background canvas}[vertical shading][bottom=blue!50,top=white!60] 
\setbeamertemplate{frametitle}[shadow]

%% ==== Presentation Options ====
\usepackage{beamerthemeshadow} %Makes a color gradient in the frame title background.
% \usepackage{beamerthemesidebar}
% \usepackage{beamerthemesplit}
% \usepackage{beamerthemebars}
% \usepackage{beamerthemeclassic}
% \usepackage{beamerthemelined}
% \usepackage{beamerthemetree}
% \usepackage{beamerinnerthemerounded}
\beamertemplatenavigationsymbolsempty %To get rid of navigation symbols.

%% ==== Themes ====
% \usetheme{boxes}
% \usetheme{default}

%% Themes with tree-likes navigation bar
% \usetheme{JuanLesPins}
% \usetheme{Montpellier}
% \usetheme{Antibes}
% \usetheme{Rochester}

%% Themes with TOC sidebar
% \usetheme{PaloAlto}
% \usetheme{Berkeley}
% \usetheme{Goettingen}
% \usetheme{Marburg}
% \usetheme{Hannover}

%% Themes with mini-frame navigation
% \usetheme{Berlin}
% \usetheme{Boadilla}
% \usetheme{Ilmenau} % +1
% \usetheme{Dresden}
% \usetheme{Darmstadt} % +1
%\usetheme{Frankfurt}
% \usetheme{Singapore}
% \usetheme{Szeged}

%% Themes with section and subsection
% \usetheme{AnnArbor}
% \usetheme{CambridgeUS}
% \usetheme{Copenhagen} % +1
% \usetheme{Luebeck}
% \usetheme{Malmoe}
% \usetheme{Warsaw}

%% ==== Colors ====
%\definecolor{mygold}{rgb}{0.85, 0.60, 0.00} 
%\setbeamercolor{math text}{fg=red!60!black}
%\setbeamercolor{normal text in math text}{fg=black}

%% ==== Color Themes ====
% \usecolortheme{albatross}
% \usecolortheme{blue}
% \usecolortheme{beetle}
% \usecolortheme{beaver}
% \usecolortheme{default}
 \usecolortheme{dolphin}
% \usecolortheme{crane}
% \usecolortheme{fly}
% \usecolortheme{seagull}
% \usecolortheme{lily}
%\usecolortheme{orchid}
%\usecolortheme{whale}
%\usecolortheme{brown} 
%\usecolortheme[named=Maroon]{structure} 

%% ==== Custom Color Options ====
%% Here is an example:
%\usecolortheme[rgb={0,0.17,0.45}]{structure}
%\setbeamertemplate{headline text}[vertical shading][bottom=red!60,top=orange!30] 
%\setbeamertemplate{footline}[page number]{}

%% ==== Inner Themes ====
%\useinnertheme[shadow]{rounded}

%% ==== Outer Themes ====
\useoutertheme{infolines}

% Many people like this style of Beamer slideshow
%\usepackage{beamerthemesplit}

% To do movies in Beamer:
%\usepackage{multimedia}
% And in the document, for example:
%\movie[height=5cm,width=6.5cm,loop]{}{move.avi}

%For using .eps graphics
%\usepackage{graphicx} 
% And in the document, for example:
% \begin{figure}
%  \includegraphics[scale=0.5]{Sines12.eps}
% \end{figure}


\title[My Short Title]{My Full-Length Title}

%\subtitle{My Subtitle}

\author[Author1, Author2]
{X.~Author1\inst{1} \and Y.~Author2\inst{2}}

\institute[Universities] % (optional, but mostly needed)
{
  \inst{1}
  Department of Mathematics\\
  University of Author1
  \and
  \inst{2}
  Department of Engineering\\
  University of Author2
}

\date[TAMU pre-REU 2012]
{Texas A\&M University pre-REU program, 2012}

% If you have a file called "university-logo-filename.xxx", where xxx
% is a graphic format that can be processed by latex or pdflatex,
% resp., then you can add a logo as follows:

% \pgfdeclareimage[height=0.5cm]{university-logo}{university-logo-filename}
% \logo{\pgfuseimage{university-logo}}

% Delete this, if you do not want the table of contents to pop up at
% the beginning of each subsection:
\AtBeginSubsection[]
{
  \begin{frame}<beamer>
    \frametitle{Outline}
    \tableofcontents[currentsection,currentsubsection]
  \end{frame}
}


% If you wish to uncover everything in a step-wise fashion, uncomment
% the following command: 

%\beamerdefaultoverlayspecification{<+->}


\begin{document}

\begin{frame}
  \titlepage
\end{frame}

\begin{frame}
  \frametitle{Outline}
  \tableofcontents
  % You might wish to add the option [pausesections]
\end{frame}

\section{Beamer Basics}

\subsection{Frames}

% ===========================================================
\begin{frame} 
  \begin{itemize}[<+->]
   \item Beamer is the LaTeX-way of making presentations (and posters).
   \item Invented by Til Tantau in 2004 (also invented tikZ in 2006).
   \item Unlike powerpoint, the finished document is a pdf file.
   \item Each slide is called a ``frame''.
   \item Each change in the frame is just a new page in the pdf file.
   \item Frames are easy to learn.
   \item Timing the frames is slightly more involved, but not too hard.
   \item Sections and subsections work just like in usual \LaTeX.
  \end{itemize}
\end{frame}
% ===========================================================

% ===========================================================
\begin{frame} 
  Documents are compiled with \texttt{pdflatex} (or \texttt{latex}), just like a usual \LaTeX document.  
  \pause
  Math works just like in \LaTeX.  \\
  \pause
  It can either be inline style: $\int_\Omega df = \oint_{\partial\Omega} f$, or in display style:
  \begin{equation*}
   \int_\Omega df = \oint_{\partial\Omega} f
  \end{equation*}
  \pause
  Note that you usually shouldn't label equations in a presentation.
\end{frame}
% ===========================================================

\section{Timing}

% ===========================================================
\begin{frame}{Pausing} 
 The first part.
 \pause
 The second part.
 \pause
  The third part.
\end{frame} 
% ===========================================================

% ===========================================================
\begin{frame}{Detailed pausing} 
 The first part.
 \pause
 The second part.
 \pause
  The third part.
\end{frame} 
% ===========================================================

% ===========================================================
\begin{frame}{Itemizing (version 1)} 
 
\begin{itemize} 
  \item Introduction 
  \pause 
  \item Statement of the main theorem 
  \pause 
  \item Technical lemmata 
  \pause 
  \item Proof of the main theorem 
  \pause 
  \item Conclusions 
\end{itemize} 
 
\end{frame} 
% ===========================================================

% ===========================================================
\begin{frame}{Itemizing (version 2)} 
 
\begin{itemize}[<+->]
  \item Introduction 
  \item Statement of the main theorem 
  \item Technical lemmata 
  \item Proof of the main theorem 
  \item Conclusions 
\end{itemize} 
 
\end{frame} 
% ===========================================================


\section{Hyperlinks}

% =========================================================
\begin{frame}[label=MySlide]{Introduction}

This slide is labeled ``MySlide''.

\end{frame}

% =========================================================
\begin{frame}{Some other slide}

If you click \hyperlink{MySlide}{here}, you will jump to the slide
labeled ``MySlide''.

\bigskip

Clicking \hyperlink{MySlide}{\beamerbutton{here}} will also
take you to the ``MySlide'' slide.

\end{frame}
% ===========================================================

% ===========================================================
\begin{frame}{Theorems and such}

\begin{definition}
  A triangle that has a right angle is called
  a \emph{right triangle}.
\end{definition}

\begin{theorem}
  In a right triangle, the square of the hypotenuse
  equals the sum of the squares of the two other sides.
\end{theorem}

\begin{proof}
  We leave the proof as an exercise to our astute reader.
  We also suggest that the reader generalize the proof to
  non-Euclidean geometries.
\end{proof}

\end{frame}
% ===========================================================

% ===========================================================
\begin{frame}
  \frametitle{Detailed Timing Commands}
  You can create overlays\dots
  \begin{itemize}
  \item using the \texttt{pause} command:
    \begin{itemize}
    \item
      First item.
      \pause
    \item    
      Second item.
    \end{itemize}
  \item
    using overlay specifications:
    \begin{itemize}
    \item<3->
      First item.
    \item<4->
      Second item.
    \end{itemize}
  \item
    using the general \texttt{uncover} command:
    \begin{itemize}
      \uncover<5->{\item
        First item.}
      \uncover<6->{\item
        Second item.}
    \end{itemize}
  \end{itemize}
\end{frame}
% ===========================================================








% ===========================================================
\begin{frame}
  \frametitle{Make Titles Informative. Use Uppercase Letters.}
  \framesubtitle{Subtitles are optional.}
  % - A title should summarize the slide in an understandable fashion
  %   for anyone how does not follow everything on the slide itself.

  \begin{itemize}
  \item
    Use \texttt{itemize} often.
  \item
    Use very short sentences or short phrases.
  \end{itemize}
\end{frame}
% ===========================================================


\section*{Summary}

\begin{frame}
  \frametitle<presentation>{Summary}

  % Keep the summary *very short*.
  \begin{itemize}
  \item
    The \alert{first main message} of your talk in one or two lines.
  \item
    The \alert{second main message} of your talk in one or two lines.
  \item
    Perhaps a \alert{third message}, but not more than that.
  \end{itemize}
  
  % The following outlook is optional.
  \vskip0pt plus.5fill
  \begin{itemize}
  \item
    Outlook
    \begin{itemize}
    \item
      Something you haven't solved.
    \item
      Something else you haven't solved.
    \end{itemize}
  \end{itemize}
\end{frame}



% All of the following is optional and typically not needed. 
\appendix
\section<presentation>*{\appendixname}
\subsection<presentation>*{For Further Reading}

\begin{frame}[allowframebreaks]
  \frametitle<presentation>{For Further Reading}
    
  \begin{thebibliography}{10}
    
  \beamertemplatebookbibitems
  % Start with overview books.

  \bibitem{Author1990}
    A.~Author.
    \newblock {\em Handbook of Everything}.
    \newblock Some Press, 1990.
 
    
  \beamertemplatearticlebibitems
  % Followed by interesting articles. Keep the list short. 

  \bibitem{Someone2000}
    S.~Someone.
    \newblock On this and that.
    \newblock {\em Journal of This and That}, 2(1):50--100,
    2000.
  \end{thebibliography}
\end{frame}

\end{document}


