\documentstyle[12pt]{amsart}
\topmargin=-.5in
\textheight=9in
\oddsidemargin=0in
\evensidemargin=0in
\textwidth=6.6in
\pagestyle{empty}
\begin{document}
\begin{center}
{\bf MATH 609-600 \\
Homework \#5 \\
Numerical  Integration \& ODE Methods}
\end{center}


\small  
\bigskip
\noindent
Solve any set of problems for 100 points.
%The homework should be presented at the beginning of the class.
5 pts per day penalty for delay of the homework applies.
 
\bigskip
\noindent


\begin{enumerate}

\item (10 pts) Show that the quadrature 
$$
 \displaystyle \int_{0}^{ \infty} e^{-x} f(x) dx \approx 
\frac{2+ \sqrt{2}}{4} f(2- \sqrt{2})+
\frac{2- \sqrt{2}}{4} f(2+ \sqrt{2})
$$
has algebraic degree of accuracy 3.


\item (10 pts) Find the nodes and the coefficients of the Gauss 
quadrature with two nodes for evaluating the integral
$$
\int_{-1}^1 \frac{f(x)}{\sqrt{1-x^2} } dx.
$$
 
\item  (20 pts)
Prove that if the interval is symmetric with respect to the
origin and if $w(x)$ is an even function, then the Gaussian 
nodes will be symmetric  respect to the
origin. So if the roots are ordered $x_0 < x_1 < \dots < x_n$, 
then $x_i=-x_{n-i}$ and $A_i=A_{n-i}$ for $i=0,1, \dots,n.$

\item (20 pts) Define the Legendre polynomial  $P_n(x)$ of degree $n$ by
$$
P_n(x)=\frac{1}{2^n ~n!}\frac{d^n(x^2-1)^n}{dx^n}, ~~ n=0,1, \dots.
$$
Show that   $P_n(x)$ has $n$ distinct zeros in the interval $(-1,1)$ which 
are symmetric with respect to the origin.

\item (20 pts) Consider the Gaussian quadrature %is written in the form
$$
\int_{-1}^{1} f(x) dx \approx  \displaystyle \sum_{i=1}^{n} A_k f(x_k).
$$
Show that 
$%$
%A_k=\frac{2}{(1-x_k^2)[P'_n(x_k)]^2}, ~~k=1,2,\dots, n.
A_k=2/[(1-x_k^2)[P'_n(x_k)]^2], ~~k=1,2,\dots, n,
$%$
 where $P_n$ is the defined above Legendre polynomial. \\
Hint: Use the fact that 
$P_n(1)=1$, $P_n(-1)=(-1)^n$ and the equality
$$
\int_{-1}^{1}P_n(x) P'_n(x)/(x-x_k) dx= A_k[P'_n(x_k)]^2.
$$

\bigskip
\noindent
Below we consider the following initial value problem: find $x(t)$ such that 
$x'(t)=f(t,x)$ for $t > t_0$ and satisfying the initial condition $x(t_0)=x_0$.
Also $\eta_n$ denotes the approximation of $x(t_n)$ by the numerical method.
\bigskip
\noindent

\item (20 pts) Consider the Runge-Kutta method 
\begin{equation}\label{RK}
\eta_0=x_0, ~~~~\eta_{i+1}= \eta_i + h \, \Phi(t_i,\eta_i;h), ~~~i=0,1,...
\end{equation}
where
$ \displaystyle \Phi(t,x;h)= \frac14 k_1(t,x) + \frac34k_2(t,x)$
with
$ k_1(t,x)=  f(t,x) $,
$ k_2(t,x) =  f(t+ \frac{2}{3}h, x+ \frac{2}{3} h k_1).$
Show that the method is of second  order.

\item (20 pts) For the above initial value problem 
consider the following (in general, implicit) 
Runge-Kutta method \eqref{RK}, where:
\begin{equation}\label{Taylor-1}
\begin{array}{rll}
 \Phi(t,x;h)& = & a_1 k_1 + a_2k_2 \\[1ex]
        k_1 & = & f(t+ \alpha_1 h, x + \beta_{11} h k_1)\\[1ex]
        k_2 & = & f(t+ \alpha_2 h, x + \beta_{21} h k_1 + \beta_{22}h k_2).
\end{array}
\end{equation}
(a) Find the conditions that the coefficients $a, \alpha, \beta $ need 
to satisfy so that the {\bf explicit} 
method (i.e. $\beta_{11} = \beta_{22}=0$) is of second order. 
Give at least one set of coefficients that satisfy these conditions.\\
(b) Find the conditions that the coefficients $a, \alpha, \beta $ need 
to satisfy so that the {\bf implicit} 
method (i.e. $\beta_{11} \not= 0$ and $\beta_{22} \not= 0$) is of second order.

\item  (20 pts)  
Derive an explicit multi-step method of order four (Adams-Bashforth four step
method) that uses integration in the interval $(t_i,t_{i+1})$. Write down the expression
for the local truncation error.

\end{enumerate}
 
 
 
 
\end{document}
 